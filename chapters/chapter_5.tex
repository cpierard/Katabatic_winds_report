The calibration done between the different type instruments showed that there was consistency between the temperature of the Thermocouples and the CSAT3's, measured at similar heights. Although, there was a significant horizontal separation between the CSAT3-B and the Thermocouple No.1 and between the CSAT3 No.4 and the Thermocouple No.4, the measurements where similar, as the high correlation value pointed out. This gave us the confidence of using the measurements of the Thermocouples to characterise the vertical profiles, with the advantage that there were more Thermocouples installed on the mast that CSAT3's to measure the temperature.

Doing the same calibration between the wind speed measured with the lowest Windsonic 2-D and the highest CSAT3, we found that the correlation was of 0.94, which shows that the wind speed of both instruments had the approximate same magnitude, even though there was a vertical separation of 0.7m between them. This is important at the moment of plotting the mean vertical wind profiles because there must be coherence in the measurements done with a different type of instruments for the profile to make sense. 

From the whole campaign, we were able to identify two nights with katabatic winds. As shown by the vertical profiles, the maximum in the mean wind speed was located between 0.5~m and 0.75~m. We expected to find this maximum at a higher level, to have two CSAT3's that could measure the wind below the maximum and two CSAT3's above the maximum, but instead of this, only one anemometer measured the wind below the maximum. 

During the night of the 23rd to the 24th, the momentum fluxes profiles showed a negative flux for the levels below the wind maximum and a positive flux above the maximum, with the exception of some rogue profiles that were negative at all levels or were negative above the wind maximum. Regarding the sensible heat flux, all the profiles were negative because of the stable environment, but there where some variations when the profiles should remain constant. For the TKE, the profiles showed a small value of TKE close to the surface, and an increase of TKE as the height increased, which correspond to what was expected. 

Globally, the campaign was a success because of the 17 days of anticyclonic conditions, which allowed to take sufficient measurements of katabatic winds structure, with better spacial resolution due to the improvements implemented based on previous campaigns. Despite that this work focused only on two nights, there are still many other nights to analyse and a lot of work ahead to do.