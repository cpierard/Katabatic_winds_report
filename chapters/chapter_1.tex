Katabatic winds or gravity currents are downslope winds that are generated when cold dense air is accelerated down a topographic slope due to surface cooling that gives the air a greater density than the free atmosphere~\citep{poulos2008observational}. They can be found in many places over the world, but they are mostly present in cold mountainous regions and over glaciers. 

Katabatic winds play an important role in the local weather and the dispersion of contaminants in the valleys. These winds can flow down the mountains and fill cool air pools that accumulate in bottom of valleys. This can result in a very cool layer of air that doesn't mix with the top layers, called temperature inversions, which have negative effects on urban valleys where pollution is present~\citep{largeron2016persistent}.

Nowadays, katabatic winds are not well represented or parameterized in mesoscale models for weather forecast because the spatial scale at which this winds occur can't be represented with those models and is a potential source of errors in weather forecasts over mountainous regions. 

There have been several studies of katabatic winds historically, mostly centred in the alpine regions of Europe, some parts of North America, Greenland and Antarctica. Despite this, there is a lack of suitable data obtained from measurements that can be used to test or adapt the theoretical models of katabatic winds~\citep{manins1979katabatic}. Also, there are few studies of these winds on steep slopes, mostly these are focused in regions where the slopes is smaller than $10 \degree$.