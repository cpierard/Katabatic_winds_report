Katabatic winds or gravity currents are downslope winds that are generated when cold dense air is accelerated down a topographic slope due to surface cooling that gives the air a greater density than the free atmosphere~\citep{poulos2008observational}. They can be found in many places over the world, but they are mostly present in cold mountainous regions and over glaciers. 

Katabatic winds play an important role in the local weather and the dispersion of contaminants in the valleys. These winds can flow down the mountains and fill the bottom of valleys with cold air. This can result in a stable stratification of the air, where the bottom air layer is colder than the upper layers, which is known as a  temperature inversion. This inversion inhibits the convection to develop during the day traps and concentrates pollutants in the bottom of the valley \citep{largeron2016persistent}.

Nowadays, the role of small scale turbulent processes over mountainous regions, like katabatic winds, are not well represented in weather forecasting and climate models. The parametrizations used in most of the models are useful for flat terrain but are too simplistic for complex terrain, which is a potential source of errors for weather forecasts \citep{serafin2018exchange}. 

There are several studies of katabatic winds, mostly centred in the alpine regions of Europe, some parts of North America, Greenland and Antarctica, but despite this, there is a lack of suitable data obtained from measurements over steep slopes (more than 10$\degree$) that can be used to test or adapted to the theoretical models of katabatic winds~\citep{manins1979katabatic}. 

%This project is based on the work done by previous students, who made their master thesis on katabatic winds with different data sets obtained from previous field campaigns. In Particular, \cite{jakob} processed the data of the 2015 campaign, that took place from 7 to 22 April 2015, during an episode of anticyclonic conditions. In his work, he had to detect the days in which the instruments measured katabatic winds. For this, he analysed the wind profiles and temperature profiles to see if they fitted the description of the katabatic winds. After selecting the days, he computed the Reynolds stress tensor, the sensible heat flux and the turbulent kinetic energy. Is important to highlight that he analysed the region where the maximum of the velocity of the jet is expected to be. As a suggestion for future projects, he points out that it could be interesting to study the turbulence beneath the wind maximum, by placing more sensors in the lower levels of the masts.

%Also, my work takes in to account the previous work done by \cite{claudine} and \cite{alban}. Both of them analysed the data from the November 2012 campaign. They followed the same methodology as \citeauthor{jakob} did, analysing the same turbulence characteristics of the jet. The main difference is that their work was focused on adapting the Prandtl analytical model with the different measurements recorded in the field. Another difference is that \citeauthor{claudine} did a spectral analysis of the data, which allowed her to observe the energy cascade. 

%Taking into account their work and the experience gained in the previous campaingns, a new campaign was planned for this year, which was held during the 12 of February to the 28 of February. There were more sonic anemometers installed in the mountain
